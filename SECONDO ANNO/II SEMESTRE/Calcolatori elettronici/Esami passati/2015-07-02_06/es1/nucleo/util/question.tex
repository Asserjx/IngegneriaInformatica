Nel nucleo che abbiamo studiato tutta la memoria, con l'esclusione
delle pile, \`e condivisa tra tutti i processi. Vogliamo aggiungere un
meccanismo tramite il quale solo alcuni processi, e non necessariamente
tutti, possano condividere della memoria.

Prevediamo quindi che un processo possa creare delle zone di memoria, dette 
\verb|shmem|, ciascuna con
un identificatore unico. I processi che voglio accedere ad una \verb|shmem|
devono aggiungerla al proprio spazio di indirizzamento, specificandone l'identificatore.
Una volta aggiunta, la \verb|shmem| sar\`a disponibile contiguamente all'interno della parte
utente/condivisa dello spazio di indirizzamento del processo. Un processo pu\`o aggiungere
pi\`u \verb|shmem| al proprio spazio e le diverse \verb|shmem| non devono sovrapporsi.
Non \`e importante che i processi che condividono una stessa \verb|shmem| la vedano tutti
allo stesso indirizzo.

Per descrivere una \verb|shmem| aggiungiamo al nucleo la seguente struttura dati:

\begin{verbatim}
    struct des_shmem {
        natl npag;
        natl first_frame_number;
    };
\end{verbatim}

Il campo \verb|npag| contiene la dimensione (in pagine) della \verb|shmem|.
Tutti i frame che contengono la \verb|shmem|, nell'ordine in cui
devono comparire nella memoria di tutti i processi che la condividono,
sono mantenuti in una lista, implementata tramite l'array \verb|vdf|. Il numero del
primo frame della lista \`e contenuto nel campo \verb|first_frame_number| e l'elmento
\verb|vdf[fn]| di ogni \verb|fn| della lista contiene il numero del frame successivo
(il valore 0 termina la lista).

Inoltre, aggiungiamo il seguente campo ai descrittori di processo:

\begin{verbatim}
    addr avail_addr;
\end{verbatim}

Questo campo contiene il primo indirizzo libero nella parte utente/condivisa del processo.
Tutti gli indirizzi da \verb|avail_addr| fino a \verb|fin_utn_c| (escluso) sono disponibili
per contenere zone di memoria condivisa.

Aggiungiamo infine le seguenti primitive:
\begin{itemize}
   \item \verb|natl shmem_create(natl npag)| (tipo 0x5c, gi\`a realizzata):
   	Crea una nuova zona di memoria condivisibile tra pi\`u processi, grande \verb|npag| pagine,
	e ne restituisce l'identificatore.
	La primitiva si limita ad allocare i frame necessari e a inserirli in una lista.
	Se non vi sono frame liberi a sufficienza, restituisce 0xFFFFFFFF.
   \item \verb|addr shmem_attach(natl id)| (tipo 0x5d, da realizzare):
   	Permette ad un processo di aggiungere la \verb|shmem| \verb|id| al proprio spazio di indirizzamento
	e ne restituisce l'indirizzo di partenza.
	Abortisce il processo se \verb|id| non corrisponde ad una \verb|shmem| esistente.
	Restituisce 0 se non \`e stato possibile aggiungere la zona (perch\'e non vi sono
	indirizzi liberi sufficienti a contenerla).
\end{itemize}

Modificare i file \verb|sistema.cpp| e \verb|sistema.S| in modo da realizzare le primitive appena descritte.

