Una {\em barriera} \`e un meccanismo di sincronizzazione tra processi che funziona nel modo seguente:
la barriera \`e normalmente chiusa; se un processo arriva alla barriera e la trova chiusa, si blocca;
la barriera si apre solo quando sono arrivati tutti i processi, che a quel punto si sbloccano;
una volta aperta e sbloccati tutti i processi, la barriera si richiude e il meccanismo si ripete.

Vogliamo realizzare il meccanismo della barriera nel nucleo, ma solo tra i processi che si sono
preventivamente registrati. I processi possono registrarsi e deregistarsi per una barriera in qualunque momento.
La deregistrazione di un processo pu\`o causare l'apertura di una barriera, se tutti gli altri processi
registrati erano gi\`a arrivati.

Per realizzare il meccanismo aggiungiamo le seguenti primitive:
\begin{itemize}
   \item \verb|void reg()| (da realizzare):
	registra il processo corrente sulla barriera. \`E un errore se un processo
	tenta di registrarsi due volte.
   \item \verb|void dereg()| (da realizzare):
	deregistra il processo corrente sulla barriera. \`E un errore se un processo
	tenta di deregistrarsi senza essere registrato.
   \item \verb|void barrier()| (da realizzare): fa giungere il processo corrente 
	alla barriera.  \`E un errore se un processo invoca questa primitiva senza essersi registrato.
\end{itemize}

Le primitive abortiscono il processo chiamante in caso di errore e tengono conto della priorit\`a tra i processi.

Modificare i file \verb|sistema.cpp| e \verb|sistema.s| in modo da realizzare le primitive appena descritte.
Pu\`o essere necessario definire nuove strutture dati e aggiungere campi al descrittore di processo.
