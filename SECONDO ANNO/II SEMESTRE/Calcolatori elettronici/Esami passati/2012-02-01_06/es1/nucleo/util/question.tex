Vogliamo aggiungere la schedulazione di tipo {\em Round Robin} al nucleo.
In questo tipo di schedulazione ogni processo pu\`o restare continuamente in esecuzione
soltanto per certo tempo massimo, dopo di che deve cedere il processore ad un altro processo
(il primo in coda pronti). 
Nel cedere il processore, il processo si reinserisce in coda pronti come ultimo.
Poich\`e nel nucleo abbiamo anche le priorit\`a, prevediamo che il processo si inserisca come ultimo
tra quelli che hanno la sua stessa priorit\`a (ma prima di quelli con priorit\`a inferiore).
In questo modo i processi  di uguale priorit\`a vengono eseguiti a rotazione.

Per realizzare questo tipo di schedulazione aggiungiamo al descrittore di processo un campo \verb|int quanto|, 
che rappresenta il numero di tick del timer che il processo ha ancora a disposizione prima di dover cedere
il processore. Il numero massimo di tick per i quali ciascun processo pu\`o restare in modo continuato in esecuzione
\`e contenuto nella costante \verb|MAX_QUANTO|.
Ogni volta che il driver del timer va in esecuzione, decrementa il campo \verb|quanto| del processo 
attualmente in esecuzione e provvede ad eseguire le azioni necessarie ad implementare quanto
sopra illustrato.

\`E possibile utilizzare la funzione \verb|des_proc* des_p(natl id)| che restituisce il puntatore
al descrittore del processo il cui id \`e passato come argomento (restituisce 0 se il processo non esiste).

Aggiungiamo infine le seguenti primitive:
\begin{itemize}
\item \verb|void abilita_rr()| (da realizzare): abilita la schedulazione Round Robin.
	Il driver del timer deve eseguire quanto sopra specificato solo se la schedulazione Round Robin \`e
	abilitata.
\item \verb|void disabilita_rr()| (da realizzare):
	Disabilita la schedulazione round robin. Se questa era abilitata provvede anche a riportare
	a \verb|MAX_QUANTO| i campi \verb|quanto| di tutti i processi esistenti nel sistema.
\end{itemize}

Modificare i file \verb|sistema.cpp| e \verb|sistema.s| in modo da realizzare le primitive e il codice mancante.
