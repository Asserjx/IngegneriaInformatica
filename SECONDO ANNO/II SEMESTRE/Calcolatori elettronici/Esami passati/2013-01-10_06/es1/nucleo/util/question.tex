Vogliamo aggiungere al nucleo una primitiva \verb|join()| tramite la quale un processo
pu\`o sospendersi in attesa della terminazione di uno (qualunque) dei processi da lui
creati tramite \verb|activate_p()|.

Per realizzare tale primitiva introduciamo la seguente struttura:

\begin{verbatim}
struct family {
    natl nchildren;
    bool orphan;
    struct des_proc *waiting;
};
\end{verbatim}

La struttura \verb|family| rappresenta una ``famiglia'' di processi, composta da un processo ``padre'' e dei processi ``figli'', creati dal padre tramite \verb|activate_p()|.
Il campo \verb|nchildren| contiene il numero di figli creati e non ancora terminati.
Il campo \verb|orhpan| vale \verb|true| se e solo se il padre \`e terminato.
Il campo \verb|waiting| \`e una coda su cui il processo padre pu\`o bloccarsi in attesa della
terminazione di uno dei suoi figli.

Ogni processo appartiene a due famiglie: quella in cui \`e un figlio e quella in cui \`e a sua volta padre. Quindi aggiungiamo due campi al descrittore di processo:

\begin{verbatim}
    family *parent;
    family *own;
\end{verbatim}

Il campo \verb|parent| punta alla famiglia in cui il processo a cui appartiene il descrittore
\`e un figlio, e il campo \verb|own| alla famiglia in cui \`e padre
(i suoi figli punteranno alla stessa struttura, tramite il loro campo \verb|parent|).
Le strutture puntate da tali campi vengono allocate e inizializzate alla creazione del processo.
Quando un processo vuole attendere che uno dei suoi figli termini si blocca sulla lista \verb|waiting|.
Quando un processo termina, controlla la lista \verb|waiting| nella
famiglia \verb|parent| e, se necessario, risveglia il processo padre.

Poich\'e le strutture \verb|family| sono condivise tra pi\`u processi
(padre e figli) che possono terminare in qualsiasi ordine,
dobbiamo porre particolare attenzione alla loro deallocazione.
Adottiamo le seguenti regole, seguite da ogni processo
alla propria terminazione:
\begin{enumerate}
\item se la famiglia \verb|own| non ha figli, dealloca la struttura, altrimenti pone \verb|orphan|
	a \verb|true|;
\item decrementa \verb|nchildren| nella famiglia \verb|parent|; se il padre non \`e ancora
  terminato non fa altro, altrimenti dealloca la struttura se non ci sono altri figli.
\end{enumerate}
{\bf Attenzione:} per motivi tecnici alcuni processi non hanno un padre. In quel caso
il puntatore \verb|parent| deve essere inizializzato a 0 e ovviamente non vanno eseguite
le azioni del punto 2.

Modificare i file \verb|sistema.cpp| e \verb|sistema.s| in modo da realizzare le primitive e il codice mancante.
